%!TEX root = ../../main.tex

\chapter{Aktueller Stand und Analyse der CO2-Runter-App}
\label{chapter:2}

Im Rahmen dieser Studienarbeit steht die Erfassung des aktuellen Zustands und die Analyse der CO2-Runter-App im Fokus. Dies ist insbesondere wichtig, um vor Beginn unserer eigentlichen Arbeit eine umfassende Bestandsaufnahme vorzunehmen und die bestehenden Probleme zu identifizieren. Zunächst vorweg sollte geklärt werden, dass häufiger der Begriff \textit{App} fallen wird. Dies soll nicht bedeuten, dass es sich hierbei um eine mobile Applikation (App) handelt. Eigentlich ist es nämlich eine Webseite und im Verlauf dieser Studienarbeit wird App und Webseite synonym verwendet. Die Startseite der CO2-Runter-App dient als Einstiegspunkt für alle NutzerInnen. Die Startseite spielt eine entscheidende Rolle, da sie den ersten Eindruck vermittelt und die NutzerInnen dazu anregen sollte, sich weiter mit der App zu beschäftigen. Da das Hauptziel der App darin besteht, die NutzerInnen dazu zu bewegen, ihren persönlichen CO2-Fußabdruck zu ermitteln, ist es von besonderer Bedeutung, dass die Webseite die NutzerInnen aktiv dazu motiviert, diese Aufgabe anzugehen. Die Abbildung \ref{fig:co2runterapp-landingpage} zeigt die Startseite der Webseite\footnote{Die Startseite ist unter folgendem Link erreichbar: https://co2runter.karlsruhe.de/ \cite{co2runterapp}}. Bei einem ersten Blick auf die Webseite fallen mehrere Unstimmigkeiten auf. Die Dropdown-Felder wirken unverhältnismäßig lang und stören die Proportionen des Layouts. Die Webseite selbst erscheint unstrukturiert und wenig organisiert, was zu einer geringen Benutzerfreundlichkeit führt. Besonders auffällig ist der Mangel an klaren Handlungsanweisungen, die den NutzerInnenn zur Interaktion und aktiven Nutzung der App ermutigen sollen.

\begin{figure}[h]
    \centering
    \includegraphics[width=0.7\textwidth]{images/02/CO2-Runter-App-Landingpage.jpeg}
    \caption{Startseite der CO2-Runter-App}
    \label{fig:co2runterapp-landingpage}
\end{figure}

Erst nach längerem Verweilen auf der Seite werden die Buttons \textit{CO2 Rechner Starten} und \textit{zum Dashboard} sichtbar. Dennoch bleibt unklar, was sich hinter dem Begriff \textit{Dashboard} verbirgt und warum es für die NutzerInnen von Interesse sein sollte. Es fehlt eine klare Aufforderung, die NutzerInnen zur aktiven Teilnahme zu bewegen und sie dazu zu inspirieren, sich eingehender mit der Webseite zu beschäftigen.

Die ideale Startseite sollte den BesucherInnen eine übersichtliche und strukturierte Präsentation bieten und sie unmittelbar dazu motivieren, ihre persönlichen CO2-Fußabdruck zu ermitteln oder die Daten der Karlsruher Stadtteile zu sehen.

Wenn die NutzerInnen sich dazu entscheiden, den Button \textit{CO2 Rechner Starten} zu betätigen, werden sie auf eine Seite weitergeleitet, die in Abbildung \ref{fig:co2runterapp-rechner} dargestellt ist.

\begin{figure}[h]
    \centering
    \includegraphics[width=1\textwidth]{images/02/CO2-Runter-App-Rechner.jpeg}
    \caption{CO2 Rechner Formular der CO2-Runter-App}
    \label{fig:co2runterapp-rechner}
\end{figure}

Auf dieser Seite wird den NutzerInnenn ein Formular präsentiert, das sie ausfüllen müssen, um ihren endgültigen CO2-Fußabdruck zu bestimmen. Anschließend können sie diese Daten einem Stadtteil von Karlsruhe zuordnen und versenden oder ohne Datenübertragung fortfahren, was sie direkt zum Dashboard führt. Die Möglichkeit, diese Entscheidung zu treffen, wird in Abbildung \ref{fig:co2runterapp-send} illustriert.

\begin{figure}[h]
    \centering
    \includegraphics[width=1\textwidth]{images/02/CO2-Runter-App-Daten-Senden.jpeg}
    \caption{CO2 Rechner Formular Daten senden der CO2-Runter-App}
    \label{fig:co2runterapp-send}
\end{figure}

Die letzte Hauptfunktion ist das Dashboard, das die übermittelten Nutzerdaten in Grafiken anzeigt. In der Kartengrafik werden die CO2-Emissionen der Stadtteile von Karlsruhe dargestellt. Diese kommen zustande, indem NutzerInnen beim Berechnen ihrer CO2-Bilanz angeben, aus welchem Stadtteil sie aus Karlsruhe kommen, wodurch sich diese Daten anpassen können. Neben der Kartengrafik gibt es noch die Charts, in denen der durchschnittliche Betrag folgenden Kategorien zugewiesen werden kann: Mobilität, Wohnen, Konsum, Ernährung und Infrastruktur. Im Gruppen-Tab sieht man wiederum, welche Gruppe wie viel CO2 verbraucht hat, und im letzten Tab, wie der Name schon verrät, wird ein Balkendiagramm angezeigt, welches die Beteiligung der NutzerInnen und in welchem Stadtteil diese wohnen. Die Abbildung \ref{fig:co2runterapp-dashboard} zeigt die Kartengrafik des Dashboards und die jeweilige Menge an CO2, die den Stadtteilen zugeordnet wurde.

\begin{figure}[h]
    \centering
    \includegraphics[width=1\textwidth]{images/02/CO2-Runter-App-Dashboard.jpeg}
    \caption{CO2 Dashboard der CO2-Runter-App}
    \label{fig:co2runterapp-dashboard}
\end{figure}

Insgesamt bietet die CO2-Runter-App eine Plattform zur Berechnung und zum Vergleich des CO2-Fußabdrucks, wobei der Fokus auf einer verbesserten Benutzerfreundlichkeit und klaren Handlungsanweisungen liegen sollte, um die aktive Beteiligung der NutzerInnen zu fördern.
Die Struktur und Anordnungen der Unterseiten sehen im Vergleich zur Startseite wesentlich strukturierter und organisierter aus, und als NutzerIn weiß man eher, was zu tun ist.

% Überleitung ins nächste Kapitel

Im nächsten Kapitel werden einige Grundlagen und Grundstrukturen definiert und erklärt, auf die diese Studienarbeit aufgebaut und das Projekt CO2-Runter weiterentwickelt wird.
Dabei geht es sowohl um Prozesse, die nichts mit den eigentlichen Programmieraufgaben zu tun haben.
Auf der anderen Seite werden aber auch wichtige Programmierbegriffe, Frameworks und Werkzeuge erklärt, die im Laufe der Studienarbeit und des Projekts eine Rolle spielen werden.