%!TEX root = ../../main.tex

\chapter{Einleitung}
\label{chapter:1}

\section{Einführung in das Thema}

Der Klimawandel wird in vielen Teilen unserer Welt in den letzten Jahren deutlich sichtbar.
Die Klimapolitik nimmt einen höheren Stellenwert in unserer Gesellschaft an, Länder und Nationen setzen sich ambitionierte Klimaziele zum Schutz der Erde, es bilden sich vermehrt Gruppierung, die gegen den Klimawandel ankämpfen. Aber vor allem ist der Begriff Klimakrise heutzutage jedem Menschen ein Begriff. Viele Städte planen deshalb Maßnahmen, um ihre BürgerInnen über ihr Konsumverhalten aufmerksam zu machen und so zum Beispiel die Emissionsproduktion einzugrenzen.  Die Stadt Karlsruhe hat deshalb in Kooperation mit dem OK Lab Karlsruhe und der Dualen Hochschule Baden-Württemberg eine CO2-Runter App veröffentlicht. Die Webseite befasst sich mit dem ökologischen Fußabdruck der BürgerInnen Karlsruhe und besteht im Grunde aus zwei Hauptseiten: dem CO2-Rechner und dem Dashboard. Da das Projekt bereits vor einigen Jahren gestartet wurde und zwischenzeitlich Verbesserungsvorschläge und verschiedenste Bugs und Fehler aufgetreten sind, soll die Arbeit an dem Projekt erneut aufgenommen werden.

\section{Ziel und Fragestellung der Arbeit}

Das Ziel dieser Studienarbeit liegt darin, die CO2-Runter-App der Stadt Karlsruhe weiterzuentwickeln. Die Weiterentwicklung der Applikation lässt sich hierbei in zwei große Themenblöcke aufteilen. Zum einen sollen vorhandene Bugs innerhalb der Applikation identifiziert und behoben werden. Da nicht alle Funktionalitäten der Webseite einwandfrei funktionieren, soll eine Komplettanalyse der Applikation durchgeführt werden und alle gefundenen Probleme gelöst werden. Auf der anderen Seite soll bei der Weiterentwicklung der CO2-Runter App die NutzerIn mehr in den Vordergrund rücken. Mithilfe einer Umfrage sollen potenzielle NutzerInnen der Webseite zu verschiedenen Kategorien befragt werden. Unter Zuhilfenahme der erhobenen Daten soll die CO2-Runter-App in einer Art weiterentwickelt werden, so dass möglichst viele Personen die Webseite nutzen und dazu angeregt werden, ihren CO2 Fußabdruck auszurechnen. Aus diesem Grund werden in der Studienarbeit und in der Umfrage folgende Fragestellungen im Vordergrund stehen:

\begin{itemize}

    \item Wie können Anreize geschaffen werden, um Personen zur Reduktion ihres CO2-Fußabdrucks zu motivieren?

    \item Welche Strategien können eingesetzt werden, um Menschen zu einem umweltfreundlichen Denken hinsichtlich des Klimawandels zu bewegen?

    \item Welches Webseitendesign ist am effektivsten in der Steigerung des Interesses an klimafreundlichem Verhalten?

\end{itemize}
