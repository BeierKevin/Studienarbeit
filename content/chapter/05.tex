%!TEX root = ../../main.tex

\chapter{Nutzerzentrierte Umfrage}
\label{chapter:5}

Während dieser Studienarbeit steht die Verbesserung und Restrukturierung der CO2-runter Webseite im Vordergrund. Dabei soll besonders auf die Nutzergruppe eingegangen werden, um mit dem bereits vorgestellten Prinzip des Nutzerzentrierten Designs die Webseite für den Endnutzer ansprechender und interessanter zu gestalten.
Durch dieses Vorgehen soll die Webseite sowohl in den Punkten Design und NUtzerfreundlichkeit, aber auch im Allgmeinen verbessert werden. Das Ziel dahinter ist, dass mehr potenzielle Nutzer gefunden und angesprochen werden, um so zu einem klimabewussteren Denken anzuregen. \\

Da der Nutzer im Zentrum des Vorgehens steht, werden selbstverständlich Ideen, Kritik und Einflüsse von potenziellen Nutzern der Webseite, aber auch von Menschen im Allgemeinen, über die bisherige Webseite benötigt. Um diese Kritik und Einflüsse zu sammeln wird im Rahmen dieser Studienarbeit eine nutzerzentrierte Umfrage durchgeführt.
Mit Hilfe der Umfrage sollen Meinungen, Kritik und Ideen von unparteiischen Personen gesammelt werden, so dass diese im Anschluss ausgewertet werden können und dem Entwicklerteam entscheidende Hinweise darüber bieten, welche Aspekte der aktuellen Webseite größeren Sanierungsbedarf haben bzw. welche Aspekte bereits ihren Zweck und Wert erfüllen.
In den folgenden Unterkapiteln wird das Vorgehen von der Sammlung der möglichen Fragen der Umfrage, bis hin zur Auswertung der Umfrageergebnisse dokumentiert.
% TODO: die vlt einarbeiten oder sind die sachen schon drin ?
% > Das dann nutzen und eine Umfrage machen wie die Webapp mehr genutzt werden kann und auch verbessert werden kann
% > wie siehts bei mir eigetlich aus aus der User perspektive
% > interaktivere Landing Page für den Nutzer
% > Anregen das er ein Forumlar ausfüllt um zu erfahren wie viel er macht im verlgiech zum avg. User so ?
% > Userzentriertes Vorgehen was ist das
% > Die Ansätze davon erleutern
% > Abwegen was genutzt wird davon
% > Was ist User Zentrietes Vorgehen und was für Verfahren gibt es um die Webseite

\section{Methodik der Umfrage}
%Eine eigene Umfrage zu erstellen und durchzuführen bedeutet viel Aufwand. Besonders die Planung der Umfrage und das Sammeln von Fragen erfordert einen höheren Zeitaufwand als eventuell erwartet.
%In den nächsten beiden Unterkapitel werden die Phasen der Umfrageplanung und -durchführung beschrieben.

Die Methodik der Umfrage bildet das methodische Grundgerüst, das die systematische Erfassung und Analyse von Daten ermöglicht, um präzise Einblicke in die untersuchte Thematik zu gewinnen. In diesem Kapitel werden die spezifischen Vorgehensweisen und Verfahren detailliert erläutert, die im Rahmen der durchgeführten Umfrage angewendet wurden.
Von der Auswahl der Stichprobe über die Gestaltung des Fragebogens bis hin zur Datenerhebung und -auswertung werden die methodischen Schritte transparent dargestellt.

\subsection{Planung der Umfrage}
Zu Beginn jeder Umfrage steht die Planung der Umfrage. Die Planung ist einer der wichtigsten, wenn nicht sogar der wichtigste, Schritt einer Umfrage, da auf Basis der in der Planung festegelegten Konzepte etc. die finale Umfrage aufbaut und schlußendlich auch durchgeführt wird.
Dazu wird im Buch "Umfrage" beschrieben, dass jedes Forschungsprojekt grundsätzlich mit einer Planungsphase beginne. Darin werden unter anderem die theoretische Fundierung der Forschungsfrage als auch der daraus resultierende Art der Froschung festgelegt./ \cite{umfrage:2011}

Die Forschungsfrage wird am Anfang eines Forschungsprojektes bestimmt und steuert das weitere Vorgehen des Projektes. \cite{umfrage:2011}
In unserem Fall bearbeiten wir die Froschungsfragen: "Wie kann man mehr Menschen dazu motivieren, die CO2-runter Webseite zu nutzen?" und "Wie kann man klimabewussteres Verhalten im allgmeinen erzeugen?"\

Mit der Forschungsfrage ergibt sich zumeist automatisch auch die Art der Forschung. Dabei unterscheidet man im Allgemeinen grundsätzlich vier Arten von Untersuchungen:

\begin{itemize}
    \item \textbf{Explorative Untersuchung:} Untersucht man einen Bereich mit seiner Untersuchung bzw. Umfrage, in welchem bisher nur sehr wenige bzw. keine Informationen und Erkenntnisse vorliegen, so handelt es sich um eine explorative Untersuchung. Das Ziel einer explorativen Untersuchung ist es, einen ersten Überblick über den Untersuchungsbereich zu gewinnen und Grundlagenforschung durchzuführen. Häufig wird diese Art der Untersuchung als Umfragenforschungsprojekt als vorbereitende Forschung verwendet. \cite{umfrage:2011}
    \item \textbf{Deskriptive Untersuchung:} Bei einer deskriptiven Untersuchung gibt es bereits ein recht großes Vorwissen. Das primäre Ziel ist hierbei Detailinformationen zu einem Thema zu erlangen. Zusätzlich im Zentrum des Interesses stehen hier Fragen nach der Verteilung bestimmter Merkmale und Merkmalskombinationen,aber auch nach Veränderungen z.B. im Zeitverlauf \cite{umfrage:2011}\ Ergebnisse einer deskriptiven Untersuchung werden höufig in Tabellen präsentiert. Bekannt sind diese aus Printmedien und/oder dem Fernsehen.\cite{umfrage:2011}
    \item \textbf{hyptothesentestende bzw. kausalanalytische Forschung:} Sollte man bei einer Untersuchung bereits vor der empirischen Untersuchung Vermutungen über die Ausprägung bestimmter Verteileungen angestellte werden, spricht man von einer hypothesentestenden Untersuchung.\cite{umfrage:2011}
\end{itemize}

Bei unserer Umfrage handelt es sich um eine kausale Untersuchung. Das liegt daran, dass mit der Umfrage eine Ursache-Wirkung-Beziehung zwischen den Aspekten der Webseite und der Nutzerfreundlichkeit herauszufinden.
Durch die Sammlung der Daten, welche Aspekte der aktuellen Webseite gut oder schlecht bewertet wurden, kann man Hypothesen aufstellen, welche Elemente der Webseite zu einer positiven bzw. negativen Nutzererfahrung führen.\

Bevor die eigentliche Umfrage erstellt wurde, wurden in einem Dokument Fragen gesammelt um einen Fragenkatalog zu erstellen. Der Fragenkatalog dient zur Sammlung möglicher Fragen, damit diese nach der Sammelphase erneut überarbeitet und gewichtet werden können.
Der Fragenkatalog besteht aus insgesamt 31 Fragen. Die Fragen wurden in einer Brainstorming-Aktion gesammelt und ohne Ordnung notiert. Nach der Mrainstorming-Aktion wurden die Fragen kategorisiert. Dabei wurden die folgenden Kategorien erstellt und für die Ordnung der Fragen genutzt:
\begin{itemize}
    \item CO2 Fußabdruck
    \item Motiviation Nutzung
    \item Verbesserung Webseite
    \item aktueller Stand der Webseite
\end{itemize}

\subsection{Durchführung der Umfrage}

\subsection{Auswahl der Zielgruppe}

\section{Erhebung von Nutzerfeedback}

\subsection{Analyse der Umfrageergebnisse}

\subsection{Identifikation von Herausforderungen und Verbesserungspotenzial}

\section{Userzentrierte Ansätze}

\subsection{Einführung in userzentriertes Vorgehen}

\subsection{Anwendung von Userzentrierten Designverfahren}

\subsection{Evaluierung der effektivsten Ansätze}
