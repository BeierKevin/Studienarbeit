%!TEX root = ../../main.tex

\chapter{Zusammenfassung und Fazit}
\label{chapter:7}

In diesem Kapitel wird eine Zusammenfassung der Erkenntnisse dieser Studienarbeit präsentiert. Es umfasst einen Rückblick auf die gewonnenen Erkenntnisse und darauf, was letztendlich implementiert wurde, sowie die Ergebnisse der Arbeit. Darüber hinaus werden Schlussfolgerungen gezogen, die nicht nur auf vergangenen Ereignissen und Erkenntnissen basieren, sondern auch einen Ausblick auf die Zukunft der CO2-Runter-App geben und Empfehlungen für deren Weiterentwicklung enthalten.

\section{Zusammenfassung der Arbeitsergebnisse}

Die Arbeitsergebnisse lassen sich grundsätzlich in zwei übergeordnete Kategorien unterteilen: einmal in die nutzerzentrierte Umfrage und dann in die darauf folgende Implementierung der erhobenen Kenntnisse mit der Neuentwicklung der CO2-Runter Webseite. Im folgenden Abschnitt wird auf beide Kategorien eingegangen und die Ergebnisse zusammengefasst.

\subsection{Nutzerzentrierte Umfrage}

Die nutzerzentrierte Umfrage war ein essenzieller Bestandteil der Studienarbeit, um Einblicke in die Bedürfnisse und Präferenzen der potenziellen NutzerInnen der CO2-Runter Webseite zu gewinnen. Die Umfrage wurde durchgeführt, um die Nutzerfreundlichkeit und das Design der Webseite zu bewerten und wertvolles Feedback zu erhalten, das bei der weiteren Entwicklung und Optimierung der Webseite berücksichtigt werden kann. Aus diesen Ergebnissen konnten letztendlich Ziele bzw. Anforderungen für die Implementierung der Webseite abgeleitet werden.

Die Umfrageergebnisse zeigten eine hohe Sensibilisierung der TeilnehmerInnen für den CO2-Fußabdruck sowie ein Interesse an einer Plattform wie CO2-Runter, die ihnen hilft, ihren Fußabdruck zu berechnen und Tipps für den Klimaschutz zu erhalten. Insbesondere wurde das Potenzial von Funktionen wie einem Punktesystem, Möglichkeiten zum Teilen von Fortschritten und Tipps zur Verhaltensänderung positiv bewertet.

Die Bewertung der Nutzerfreundlichkeit und des Designs lieferten wertvolle Einblicke, die bei der weiteren Optimierung der Webseite berücksichtigt werden können. Es wurde deutlich, dass eine einfache Benutzeroberfläche sowie ansprechende visuelle Elemente wichtige Faktoren für eine positive Nutzererfahrung sind.

Abschließend können die Ergebnisse der nutzerzentrierten Umfrage als Leitfaden genutzt werden, um Personas zu erstellen, die als Grundlage dienten, um die Anforderungen für die Implementierung der Webseite herauszuarbeiten und zu definieren. Diese konnten dann genutzt werden, um bei der Neuentwicklung der Webseite wichtige Aspekte, die für die NutzerInnen wichtig sind, zu berücksichtigen und umzusetzen, um eine bessere Nutzererfahrung zu gewährleisten.

\subsection{Implementierung der CO2-Runter Webseite}

Die Implementierung der CO2-Runter Webseite war der zweite Kernaspekt dieser Studienarbeit. Die Implementierung basierte auf den Ergebnissen, die aus der \hyperref[chapter:4]{nutzerzentrierten Umfrage} gewonnen wurden, und den definierten Anforderungen, die in \hyperref[chapter:5]{Kapitel 5} festgelegt wurden.

Eine effektive Methode, um die Implementierungsleistung und die Ergebnisse der finalen Webseite zu bewerten und zu messen, besteht darin, die in \hyperref[chapter:5]{Kapitel 5} definierten Anforderungen zu betrachten. Diese Anforderungen wurden durch verschiedene Kategorien sowie vorangegangene Recherchen und \acs{UCD}/\acs{HCD}-Analysen definiert. Auf Basis dieser Erkenntnisse haben wir insgesamt \textbf{19} Anforderungen identifiziert.

Ein einfacher Ansatz besteht darin zu ermitteln, wie viele der Anforderungen letztendlich implementiert wurden und welche genau. Von den \textbf{19} Anforderungen wurden \textbf{17} erfolgreich umgesetzt und implementiert. Die beiden nicht implementierten Anforderungen sind \textbf{R08} und \textbf{R13}. Diese Quote ist bereits sehr akzeptabel. Zudem ist zu beachten, dass beide nicht erfüllten Anforderungen der Kategorie der \textbf{Kann-Anforderungen} angehören, was ihre Nicht-Implementierung als nicht essentiell kennzeichnet. Eine Übersicht der implementierten Anforderungen findet sich in Tabelle \ref{table:completet-tasks}.

\begin{longtable}{|c|l|c|c|}
    \hline
    \textbf{ID}          &
    \textbf{Anforderung} &
    \textbf{Implementiert}                                                                               \\ \hline
    \endfirsthead

    \hline
    \textbf{ID}          & \textbf{Anforderung}                                 & \textbf{Implementiert} \\
    \hline
    \endhead

    \hline
    \multicolumn{3}{|r|}{{Die Fortsetzung erfolgt auf der nachfolgenden Seite}}                          \\
    \hline
    \endfoot

    \endlastfoot

    R01                  & Abrufen der Bestandsdaten von der Datenbank          & \checkmark             \\
    \hline
    R02                  & Berechnung des CO2-Fußabdrucks                       & \checkmark             \\
    \hline
    R03                  & Hochladen von CO2-Daten                              & \checkmark             \\
    \hline
    R04                  & Homepage                                             & \checkmark             \\
    \hline
    R05                  & Dashboard                                            & \checkmark             \\
    \hline
    R06                  & Gruppensystemintegration                             & \checkmark             \\
    \hline
    R07                  & FAQ- und Literaturlisten-Seite                       & \checkmark             \\
    \hline
    R08                  & Tipps und Tricks zur Reduzierung des CO2-Fußabdrucks & \textcolor{red}{X}     \\
    \hline
    R09                  & Durchführung des CO2-Rechner-Workflows               & \checkmark             \\
    \hline
    R10                  & Integration von Grafiken                             & \checkmark             \\
    \hline
    R11                  & Visualisierung und Vergleich des CO2-Fußabdrucks     & \checkmark             \\
    \hline
    R12                  & Modernisierung des Designs                           & \checkmark             \\
    \hline
    R13                  & Anzeige von weiteren Quellen/Infos im CO2-Rechner    & \textcolor{red}{X}     \\
    \hline
    R14                  & Anpassung von Question.json                          & \checkmark             \\
    \hline
    R15                  & Benutzerfreundlichkeit                               & \checkmark             \\
    \hline
    R16                  & Modernes Design                                      & \checkmark             \\
    \hline
    R17                  & Zuverlässigkeit                                      & \checkmark             \\
    \hline
    R18                  & Wartbarkeit und Erweiterbarkeit                      & \checkmark             \\
    \hline
    R19                  & Codequalität und Styleguide-Einhaltung               & \checkmark             \\
    \hline
    \caption{Übersicht der implementierten funktionalen und nicht-funktionalen Anforderungen}
    \label{table:completet-tasks}
\end{longtable}

\section{Empfehlung an die Laufzeitumgebung}
Das neue Frontend der CO2-Runter-Webseite wurde zu Testzwecken auf einem Webserver deployt und alle Funktionalitäten getestet.
Beim laufenden Betriebssystem des Webservers handelt es sich um das Linux-Betriebssystem \textbf{Ubuntu 22.04.2 LTS}.
Das Betriebssystem wurde 2020 veröffentlicht und wurde als \textit{Long-term support} gekennzeichnet.
Das bedeutet, dass es diese Ubuntuversion bis 2025 weiter gewartet und Instand gehalten wird.

Aus diesem Grund empfiehlt das Entwicklerteam, die neue Version der CO2-Runter-Webseite auf einem Webserver mit dem Betriebssystem Ubuntu 22.04 zu deployen.

\section{Schlussfolgerungen und Ausblick}

Da es sich hierbei um eine Studienarbeit mit einem begrenzten Zeitrahmen handelt, war es unser Fokus, die wesentlichen Aspekte der CO2-Runter-App zu entwickeln. Wir haben erfolgreich die Grundfunktionalitäten implementiert, um die alte Webseite durch unsere neu entwickelte zu ersetzen. In dem verfügbaren Zeitrahmen sind wir zu dem Schluss gekommen, dass wir unsere Ziele erreicht haben und sind sehr zufrieden mit dem, was wir erreicht haben und wie viel wir dabei gelernt haben.

Obwohl der Weg nicht immer einfach war und wir auf viele Herausforderungen gestoßen sind, war eine besonders herausfordernde Aufgabe die Implementierung des CO2-Rechners. Wir hatten erwartet, dass wir viele Aspekte aus dem vorherigen Rechner übernehmen könnten, was sich jedoch als schwieriger erwies als gedacht. Insbesondere stellte sich heraus, dass der Rechner im detaillierten Modus nicht ordnungsgemäß funktionierte, einschließlich der Berechnung des Konsums und der damit verbundenen Formeln. Dies zwang uns, umfassende Anpassungen vorzunehmen, was zu einer gründlichen Überprüfung und Analyse führte.

Durch umfangreiche Recherchen und Analysen gelang es uns schließlich, den Rechner zu implementieren und logischere Schlussfolgerungen zu ziehen. Im alten Rechner war es beispielsweise möglich, die Annahme zu treffen, dass bei einer bestimmten Ernährungsweise kein CO2-Ausstoß entsteht, was jedoch unrealistisch ist. Unsere Anpassungen führen zu Ergebnissen, die realistischer sind, obwohl sie aufgrund vieler Mittelwertbildungen nicht zu 100\% genau sind, aber sehr nahe am Original liegen.

Wir hoffen, dass unsere Arbeit dazu beiträgt, die Welt ein Stückchen besser zu machen. Das Projekt ist damit jedoch hoffentlich nicht abgeschlossen. Wir hoffen, dass zukünftige StudentInnen an diesem Projekt teilnehmen und es weiterentwickeln und verbessern können, da sicherlich noch viele unentdeckte Möglichkeiten bestehen. Wir haben einige Empfehlungen und Vorschläge für die Zukunft des Projekts.

\section{Empfehlungen für die Zukunft der CO2-Runter-App}

Von Anfang an war es unser Ziel, die CO2-Runter-App so zu gestalten, dass sie kontinuierlich weiterentwickelt und optimiert werden kann. Die Architektur der Webseite wurde bewusst auf Flexibilität und Erweiterbarkeit ausgelegt. Wir haben \acl{TS} als Hauptentwicklungssprache gewählt und die Anwendungsstruktur entsprechend angepasst, um zukünftige Erweiterungen nahtlos zu ermöglichen. Dadurch besteht definitiv das Potenzial, die Webseite weiterzuentwickeln, indem beispielsweise weitere Funktionen und Features durch zusätzliche Recherchen und Analysen hinzugefügt werden oder Features, die wir aufgrund von Zeitmangel nicht implementieren konnten.

Bei der Neuentwicklung der Webseite stießen wir auf Herausforderungen im Umgang mit der \acs{API}. Obwohl sie funktioniert, ist sie ausschließlich in \acl{JS} geschrieben, was zu Problemen bei der Typisierung und Fehlerbehandlung führte. Zudem war es schwierig zu verstehen, welche Daten der Endpunkt tatsächlich benötigt und zurückgibt. Für das Frontend haben wir deshalb zahlreiche Interfaces erstellt, um die Daten zu typisieren. Eine Verbesserung der \acs{API} wäre durch eine Neuschreibung in \acl{TS} möglich. Außerdem ist uns aufgefallen, dass die \acs{API} keine Möglichkeit bietet, dass NutzerInnen aus einer Gruppe austreten können. Nur der Gruppenadministrator kann NutzerInnen entfernen, indem er die Gruppe auflöst. Dies ist ein weiterer Punkt, der verbessert werden könnte. Dabei sollte auch darüber nachgedacht werden, welche weiteren Funktionen im Gruppensystem gewünscht sind, was ein hohes Potenzial für die EndnutzerInnen der Webseite darstellt. Als letztes gab es noch die konkrete Implementierung der Anforderungen \textbf{R08} und \textbf{R13}, welche leider nicht umgesetzt werden konnten. Diese Anforderungen können ein hohes Potential bieten und als sehr praktische Erweiterung des CO2-Rechners dienen und kann darauf auf der FAQ- und Literaturlisten-Seite eingebunden werden, welche mit der \textbf{R07} bereits implementiert wurde.

% Finale ableitung

Zusammenfassend ist die CO2-Runter-App ein Projekt, das uns sehr am Herzen liegt und das wir gerne weiterentwickeln und verbessern möchten. Wir sind stolz auf das bisher Erreichte und freuen uns auf die Zukunft der Webseite. Unsere Hoffnung ist es, dass die Webseite dazu beiträgt, das Bewusstsein für den CO2-Fußabdruck zu schärfen und Menschen dazu inspiriert, ihren Beitrag zum Klimaschutz zu leisten. Wir sind zuversichtlich, dass die CO2-Runter-App ein nützliches Werkzeug für alle sein wird, die ihren CO2-Fußabdruck reduzieren möchten, und dass sie dazu beitragen wird, die Welt zu einem besseren Ort zu machen.
