%!TEX root = ../../main.tex

\chapter{Zusammenfassung und Fazit }
\label{chapter:7}

\section{Zusammenfassung der Arbeitsergebnisse}

Um nun rückblickend auf die Arbeitsergebnisse zu schauen, wird nun geschaut ob und welche von den zuvor definierten Anforderungen letztlicher erfüllt werden konnte und welche nicht und was die Gründe dafür sind.

% TODO: Schauen was davon wirklich implementiert wurde und was nicht

\begin{longtable}{|c|l|c|c|}

    \hline
    \textbf{ID}          &
    \textbf{Anforderung} &
    \textbf{Implementiert}                                                                 \\ \hline
    \endfirsthead

    \hline
    \textbf{ID}          &
    \textbf{Anforderung} &
    \textbf{Implementiert}                                                                 \\ \hline
    \endhead

    \hline
    \multicolumn{3}{|r|}{{Die Fortsetzung erfolgt auf der nachfolgenden Seite}}            \\ \hline
    \endfoot

    \endlastfoot

    R01                  & Abrufen der Bestandsdaten von der Datenbank        & \checkmark \\ \hline
    R02                  & Berechnung von CO2 Fußabdruck                      & \checkmark \\ \hline
    R03                  & Hochladen von CO2 Daten                            & X          \\ \hline
    R04                  & Homepage                                           & \checkmark \\ \hline
    R05                  & Dashboard                                          & \checkmark \\ \hline
    R06                  & Tipps und Tricks zur Reduktion des CO2-Fußabdrucks & X          \\ \hline
    R07                  & Durchführung des CO2-Rechner-Workflows             & \checkmark \\ \hline
    R08                  & Integration von Grafiken                           & \checkmark \\ \hline
    R09                  & Visualisierung und Vergleich des CO2-Fußabdrucks   & \checkmark \\ \hline
    R10                  & Modernisierung des Designs                         & \checkmark \\ \hline
    R11                  & Anzeige von weiteren Quellen/Infos im CO2 Rechner  & X          \\ \hline
    R12                  & Benutzerfreundlichkeit                             & \checkmark \\ \hline
    R13                  & Modernes Design                                    & \checkmark \\ \hline
    R14                  & Zuverlässigkeit                                    & \checkmark \\ \hline
    R15                  & Wartbarkeit und Erweiterbarkeit                    & \checkmark \\ \hline
    R16                  & Code Qualität und Styleguide Einhaltung            & \checkmark \\ \hline
    R17                  & Gruppensystem einbindung                           & \checkmark \\ \hline
    R18                  & FAQ und Literaturlisten Seite                      & \checkmark \\ \hline
    \caption{Übersicht der Erledigten funktionalen und nicht-funktionalen Anforderungen}
    \\
\end{longtable}

\section{Schlussfolgerungen und Ausblick }

\section{Empfehlungen für die Zukunft der CO2-Runter-App}

Von Beginn an war es unser Ziel, die CO2-Runter-App so zu konzipieren und zu entwickeln, dass sie kontinuierlich weiterentwickelt und optimiert werden kann. Die Architektur der App wurde bewusst auf Flexibilität und Erweiterbarkeit ausgelegt. Wir haben TypeScript als grundlegende Entwicklungssprache gewählt und die Struktur der Anwendung entsprechend gestaltet, um zukünftige Erweiterungen nahtlos zu ermöglichen.

Bei der Neugestaltung des Frontends wurden moderne Praktiken angewendet, die auch in Zukunft auf das Backend (\acs{API}) übertragen und mit TypeScript weiterentwickelt werden sollen. Insbesondere sollten die bisherigen praktischen Erfahrungen mit Vuetify und dem Styling im Frontend nicht vernachlässigt werden. Dieser Ansatz zielt darauf ab, die Wartbarkeit und Erweiterbarkeit der App zu verbessern, insbesondere durch eine konsequente Typisierung und effiziente Aufrufmöglichkeiten der \acs{API}-Endpunkte.

% TODO: Empfehlungen für die Zukunft der CO2-Runter-App wäre den Backend Server neu zu schreiben das er auch in TypeScript geschrieben ist und die Datenbankanbindung zu verbessern. Bessere Fehlerbehandlung in Front und Backend
