%!TEX root = ../../main.tex

\chapter{Identifikation und Beheben bestehender Probleme}
\label{chapter:4}

Bevor es ans eigentliche geht sollte im bestehenden System die Probleme identifiziert werden und lösungen auf diese gefunden werden bevor auf Neuerungen eingegangen werden!

% TODO: die vlt einarbeiten oder sind die sachen schon drin ?
% > Wo sind die Probleme  (mit react gefixed etc.-)
% > Bugs fixen dann (dabei auf die Probleme eingehen) (Technische Probleme fixen)

% TODO: Die Fehler müssen uns zuerst gesagt werden etc --> dann erst analysieren und beheben

% TODO: Die ganzen Probleme in einer Tabelle anzeigen, die kann man dann wiederum im Fazit referenzieren um zu schauen was behoben wurde und was nicht so gut gelöst wurde

BI = Bug ID

\section{Analyse und lösen bestehender Probleme}

\subsection{[BID1] Titel}

\subsubsection{Erklärung}

\subsubsection{Lösung}

\subsection{[BIDX] ...}

\section{Übersicht der Probleme und Bugs}

\begin{longtable}{@{\extracolsep{\fill}}|c|c|c|c|@{}}
    \hline
    \multicolumn{1}{|c|}{\textbf{ID}}       &
    \multicolumn{1}{c|}{\textbf{Title}}     &
    \multicolumn{1}{c|}{\textbf{Erklärung}} &
    \multicolumn{1}{c|}{\textbf{Lösung}}                                                   \\ \hline
    \endfirsthead
    \hline
    \multicolumn{1}{|c|}{\textbf{ID}}       &
    \multicolumn{1}{c|}{\textbf{Titel}}     &
    \multicolumn{1}{c|}{\textbf{Erklärung}} &
    \multicolumn{1}{c|}{\textbf{Lösung}}                                                   \\ \hline
    \endhead

    \hline
    \multicolumn{4}{|r|}{{Die Fortsetzung erfolgt auf der nachfolgenden Seite}}            \\ \hline
    \endfoot

    \endlastfoot
    BI1                                     & Titel & Erklärungs Kapitel & Lösungs Kapitel \\ \hline
    \caption{Tabelle mit all den Problemen und Bugs}
    \\
\end{longtable}
